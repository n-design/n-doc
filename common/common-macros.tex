% General macros for the titlepage
\newcommand{\documentdate}[1]{\directlua{getDocumentDate('#1')}}
\newcommand{\documentversion}[1]{\directlua{getDocumentVersion('#1')}}
\newcommand{\nosnapshot}[1]{\directlua{tex.sprint(get_version_number_for_reflist('#1'))}}

% Als Erinnerung für später zu erledigende Aufgaben
% Markiert den übergebenen Text deutlich und setzt
% einen Merker in die Randspalte.
% define list of To Do items

% \todo creates a "to do" item. Add the item to the To Do list.  Paramter is the
% text of the to do item. Optional parameter is the text printed in the "to do"
% items list.
\DeclareNewTOC[listname={To Do}]{todo}
\newcommand{\todocolor}{magenta}
\newcommand{\todo}[2][]{\marginpar{\textcolor{\todocolor}%
    {\fbox{\textbf{To Do}}}}\textcolor{\todocolor}{#2}
  \ifthenelse{\equal{#1}{}}{}{\addcontentsline{todo}{section}{#1}}}

% Denotes an empty section. Used to keep congruence with PP outline and section
% numbers.
\newcommand{\intentionallyleftblank}{(This section intentionally left blank.)}

% Macros for Java (and other) source code
\newcommand{\code}[1]{\texttt{#1}}
% optional argument for package path is printed in a slightly smaller font
\newcommand{\java}[2][]{\textsmaller[2]{\code{#1}}\code{#2}}
\newcommand{\bundle}[1]{\footnotesize{\java{#1}}}

% Macro for RFC. Argument is the RFC's number. When "c" is used as optional
% argument, additionally a citation is printed. Note: The blank before \autocite
% is important.
\newcommand{\rfc}[2][false]{RFC\,#2%
  \ifthenelse{\equal{#1}{c}}{ \autocite{rfc#2}}{}}

% Removes underscores from the argument
\newcommand{\nounderscore}[1]{\directlua{replaceUnderscore('#1')}}
\newcommand{\nohyphen}[1]{\directlua{remove_smart_hyphen([[#1]])}}

% Macro for keywords
\newcommand{\keyword}[1]{\textsmaller[0.8]{\ccfont\textit{\directlua{replaceUnderscore('#1')}}}}

% Macro for printing error messages defined in common/db/errors.csv
\newcommand{\error}[1]{#1 -- \enquote{\directlua{getError('#1')}}}

% Macro for file names
\newcommand{\filename}[1]{{\narrowfilefont #1}}

% Macros for headlines. Creates a hypertext anchor in addition to the \label

% These versions of the command can be called with a star "*", so that headlines
% without numberings are created. Only use these macros. They take two mandatory
% arguments: First the name of the label, second the name of the headline.

\makeatletter
\newcommand{\hrefchapter}{%
  \@ifstar
  {\hrefchapterStar}%
  {\hrefchapterNoStar}}
\newcommand{\hrefsection}{%
  \@ifstar
  {\hrefsectionStar}%
  {\hrefsectionNoStar}}
\newcommand{\hrefsubsection}{%
  \@ifstar
  {\hrefsubsectionStar}%
  {\hrefsubsectionNoStar}}
\newcommand{\hrefsubsubsection}{%
  \@ifstar
  {\hrefsubsubsectionStar}%
  {\hrefsubsubsectionNoStar}}
\newcommand{\hrefparagraph}{%
  \@ifstar
  {\hrefparagraphStar}%
  {\hrefparagraphNoStar}}
\newcommand{\hrefsubparagraph}{%
  \@ifstar
  {\hrefsubparagraphStar}%
  {\hrefsubparagraphNoStar}}
\makeatother


%% Not to be called by the user, see section above
\newcommand{\hrefchapterNoStar}[2]{\hypertarget{#1}{\chapter{#2}\label{#1}}}
\newcommand{\hrefsectionNoStar}[2]{\hypertarget{#1}{\section{#2}\label{#1}}}
\newcommand{\hrefsubsectionNoStar}[2]{\hypertarget{#1}{\subsection{#2}\label{#1}}}
\newcommand{\hrefsubsubsectionNoStar}[2]{\hypertarget{#1}{\subsubsection{#2}\label{#1}}}
\newcommand{\hrefparagraphNoStar}[2]{\hypertarget{#1}{\paragraph{#2}\label{#1}}}
\newcommand{\hrefsubparagraphNoStar}[2]{\hypertarget{#1}{\subparagraph{#2}\label{#1}}}
\newcommand{\hrefchapterStar}[2]{\hypertarget{#1}{\chapter*{#2}\label{#1}}}
\newcommand{\hrefsectionStar}[2]{\hypertarget{#1}{\section*{#2}\label{#1}}}
\newcommand{\hrefsubsectionStar}[2]{\hypertarget{#1}{\subsection*{#2}\label{#1}}}
\newcommand{\hrefsubsubsectionStar}[2]{\hypertarget{#1}{\subsubsection*{#2}\label{#1}}}
\newcommand{\hrefparagraphStar}[2]{\hypertarget{#1}{\paragraph*{#2}\label{#1}}}
\newcommand{\hrefsubparagraphStar}[2]{\hypertarget{#1}{\subparagraph*{#2}\label{#1}}}
%% Cleared. End of macros not to be called by the user



% Macros for security related items.
\newcommand{\secitemfont}[1]{{\ccfont #1}}
\newcommand{\secitemformat}[1]{{\textsmaller[0.9]{\secitemfont{#1}}}}
\newcommand{\secitem}[1]{\secitemformat{\nounderscore{#1}}}

%% ------------------------------------------------------------
%%
%% Macros for table entries
%% 
\newcommand{\tcheck}{\textsmaller[1]{\checkmark{}}}
\newcommand{\tno}{\ldotp{}}
\newcommand{\toptio}{\textcolor{blue}{\textbullet{}}}
\newcommand{\tdrop}{\textcolor{blue}{\textopenbullet{}}}
\newcommand{\tadded}{\textcolor{magenta}{\tcheck{}}}
\newcommand{\leftblank}{---}

%% ------------------------------------------------------------
%%
%% Macros for
%% IDs of TLS connections
%% used in common/tls_connections.tex
%%
\newcounter{tlsconnid}
\renewcommand*\thetlsconnid{TLS.\arabic{tlsconnid}}
\newcommand{\tlsconntablerow}[2]{\luadirect{getTlsConnectionTableRow('#1', '#2')}}
\newcommand{\tlsid}[1]{\refstepcounter{tlsconnid}\secitemformat{\thetlsconnid}\label{#1}}

% Column definitions for tables, multi-line text with forced line breaks
% See https://tex.stackexchange.com/questions/12703/
\newcolumntype{L}[1]{>{\raggedright\let\newline\\\arraybackslash\hspace{0pt}}p{#1}}
\newcolumntype{C}[1]{>{\centering\let\newline\\\arraybackslash\hspace{0pt}}p{#1}}
\newcolumntype{R}[1]{>{\raggedleft\let\newline\\\arraybackslash\hspace{0pt}}p{#1}}
\newcolumntype{Y}{>{\raggedright\let\newline\\\arraybackslash\hspace{0pt}}X}
\newcolumntype{Q}[2]{>{\adjustbox{angle=#1,lap=\width-(#2)}\bgroup}l<{\egroup}}
\newcommand*\rot{\multicolumn{1}{Q{90}{0em}}}% no optional argument here, please!

\newcommand{\landscapetable}[1]{%
\afterpage{%
  \clearpage% Flush earlier floats (otherwise order might not be correct)
  \begin{landscape}% Landscape page
    \centering % Center table
    {
      \input{#1}
    }
  \end{landscape}
  \clearpage% Flush page
}%
}

% Ellipses between first and second column of tables
\makeatletter
\newcommand \mydotfill {\leavevmode \cleaders \hb@xt@ .66em{\hss .\hss }\hfill \kern \z@}
\makeatother
\newcommand{\tablekeyword}[1]{\keyword{#1}\mydotfill}
\newcommand{\defaultvalue}[1]{\textsl{#1}}

%%% Local Variables:
%%% mode: latex
%%% TeX-master: shared
%%% TeX-engine: luatex
%%% End:
