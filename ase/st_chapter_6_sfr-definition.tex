%%%%%%%%%%%%%%%%%%%%%%%%%%%%%%%%%
%% Chapter 6 : SFR Definitions %%
%%%%%%%%%%%%%%%%%%%%%%%%%%%%%%%%%

\hrefchapter{sfr}{Sicherheitsanforderungen}

\hrefsection{sfr.intro}{Hinweise und Definitionen}

Der größte Teil der Sicherheitsanforderungen wird ohne Anpassungen aus dem
Schutzprofil übernommen. Anpassungen werden kenntlich gemacht. Bei denjenigen
SFR, die das Schutzprofil bereits vorsieht, wird in diesem Security Target
darauf verzichtet, die Hierarchie der Komponenten sowie deren Abhängigkeiten zu
wiederholen. Diese Informationen sind dem Schutzprofil \citepp{} zu
entnehmen. Bei Sicherheitsanforderungen, die durch das Security Target
hinzugefügt werden, sind die Hierarchie- und Abhängigkeitsinformationen
aufgeführt.

\hrefsubsection{sfr.intro.notc}{Hinweise zur Notation}

Die typographischen Auszeichnungen für die Operationen an den SFR sind in
\tableref{tab:sfr.intro.notc} beschrieben. ST-seitige Löschungen werden immer von
einem Hinweis begleitet, wie die Löschung motiviert ist.

\begin{table}[htb]
  \centering{}
  \begin{tabularx}{\textwidth}{@{}llX@{}}
    \toprule
    Quelle & Art der Anpassung & Typographische Eigenschaften\\
    \midrule
    PP & Zuweisung (Assignment) & Zuweisungen sind \ppassigned{unterstrichen} gesetzt.\\
    & Auswahl (Selection) & Auswahlen sind \ppselected{kursiv und unterstrichen} gesetzt.\\
    & Verfeinerung (Refinement) & Verfeinerungen sind \pprefined{fett} gesetzt.\\
    & Löschung (Deletion) & Löschungen sind \ppdeleted{fett und durchgestrichen} gesetzt.\\
    \midrule
    ST & Zuweisung (Assignment) & Zuweisungen sind \stassigned{in blauer Schrift} gesetzt.\\
    & Auswahl (Selection) & Auswahlen sind \stselected{in blauer Schrift und kursiv} gesetzt.\\
    & Verfeinerung (Refinement) & Verfeinerungen sind \strefined{in blauer Schrift und fett} gesetzt.\\
    & Löschung (Deletion) & Löschungen sind \stdeleted{in blauer Schrift, fett und durchgestrichen} gesetzt.\\
    \bottomrule
  \end{tabularx}
    \caption{Typographische Konventionen}
    \label{tab:sfr.intro.notc}
\end{table}

\hrefsubsection{sfr.intro.model}{Modellierung von Subjekten, Objekten,
  Attributen und Operationen}

Die Modellierungen des Schutzprofils \citepp{} gelten auch für dieses Security
Target. Über das Schutzprofil hinaus werden die Subjekte, Objekte und Attribute
in \tableref{tab:sfr.intro.model} angenommen.

\begin{table}[htb]
  \centering{}
  \begin{tabularx}{\textwidth}{@{}lXX@{}}
    \toprule
    Subjekt & Bechreibung & Attribut \\
    \midrule
    \hypertarget{}{\subjobj{s_admin}} & \subjobjtext{s_admin} & (Kein) \\
    \subjobj{s_zeitdienst} & \subjobjtext{s_zeitdienst} & (Kein) \\
    \bottomrule
  \end{tabularx}
    \caption{Subjekte}
    \label{tab:sfr.intro.model}
\end{table}

\hrefsection{sfr.def}{Security Functional Requirements}

\hrefsubsection{sfr.def.vpnclient}{VPN Client}

\sfrsubsection{ftp_itc.1/vpn}
\sfrdefinition{ftp_itc.1.1/vpn}{\sfrunmodifiedfrompp{6.2.1}}
\sfrdefinition{ftp_itc.1.2/vpn}{\sfrunmodifiedfrompp{6.2.1}}
\sfrdefinition{ftp_itc.1.3/vpn}{\sfrunmodifiedfrompp{6.2.1}}

\hrefsubsection{sfr.def.net}{Network Services}

\sfrsubsection{fpt_stm.1}

\sfrdefinition{fpt_stm.1.1}{\sfrunmodifiedfrompp{6.2.3}}

\refinement{Reliability of time stamps is achieved by synchronisation of the
  real time clock (\objlink{oe.realtimeclock}) with time servers using the NTPv4
  protocol \cite{rfc5905}. The TOE uses the reliable time stamps for itself.}

\sfrsubsection{fpt_tdc.1/zert}
\sfrdefinition{fpt_tdc.1.1/zert}{\sfrunmodifiedfrompp{6.2.3}}
\sfrdefinition{fpt_tdc.1.2/zert}{The TSF shall use \ppselected{interpretation
    rules} when interpreting the TSF data from another trusted IT
  product.\par{}\strefined{The interpretation rules are defined in \dots{}}}

\hrefsubsection{sfr.def.packetinspection}{Stateful Packet Inspection}

\intentionallyleftblank{}

\hrefsubsection{sfr.def.selfprot}{Self Protection}

\sfrsubsection{fdp_rip.1}

\sfrdefinition{fdp_rip.1.1}{The TSF shall ensure that any previous information
  content of a resource is made unavailable upon the \ppselected{deallocation of
    the resource from} the following objects: \ppassigned{cryptographic keys
    (and session keys) used for the VPN or for TLS-connections},
  \stassigned[list of objects]{no other objects}.}

\refinement{Sensitive data muse be overwritten with constant or random values as
  soon as they are not in use anymore.\par
  \strefined{These sensitive objects are overwritten with constant values.}}

\sfrsubsection{fpt_tst.1}

\sfrdefinition{fpt_tst.1.1}{The TSF shall run a suite of self tests
  \stselected[during initial start-up, periodically during normal
  operation, at the request of the authorised user, at the conditions
  {[}assignment: conditions under which self test should
  occur{]}]{during initial start-up, at the request of the authorised
    user} to demonstrate the correct operation of
  \stselected[{[}assignment: parts of TSF{]}, the TSF]{stored TSF
    executable code}.}

\sfrdefinition{fpt_tst.1.2}{The TSF shall provide authorised users
  with the capability to verify the integrity of \ppassigned{TSF
    data}.}

\sfrdefinition{fpt_tst.1.3}{The TSF shall provide authorised users with the
  capability to verify the integrity of \stselected[{[}assignment: parts of
  TSF{]}, the TSF]{the TSF}.}


\sfrsubsection{ftp_trp.1/admin}

\sfrdefinition{ftp_trp.1.1/admin}{The TSF shall provide a
  communication path between itself and \stselected[remote,
  local]{local} users that is logically distinct from other
  communication paths and provides assured identification of its end
  points and protection of the communicated data from
  \stselected[modification, disclosure, {[}assignment: other types of
  integrity or confidentiality violation{]}]{modification,
    disclosure}.}

\sfrdefinition{ftp_trp.1.2/admin}{The TSF shall permit \stselected[the TSF,
  local users, remote users]{local users} to initiate communication via the
  trusted path.}

\sfrdefinition{ftp_trp.1.3/admin}{\sfrunmodifiedfrompp{6.2.6}}

\applicationnote{\sfr{ftp_trp.1/admin}}{\label{appnote:clientcert} The TOE
  supports mutual authentication with certificates in the TLS handshake.}


\hrefsubsection{sfr.def.crypto}{Cryptographic Services}

\sfrsubsection{fcs_cop.1/hash}

\sfrdefinition{fcs_cop.1.1/hash}{The TSF shall perform
  \ppassigned{hash value calculation} in accordance with a specified
  cryptographic algorithm \ppassigned{\stdeleted{SHA-1,} SHA-256,}
  \stassigned[list of SHA-2 Algorithms with more than 256 bit
  size]{SHA-512} and cryptographic key sizes \ppassigned{none} that
  meet the following: \ppassigned{FIPS PUB 180-4
    \autocite{FIPS180-4}}.}

\sfrsubsection{fcs_cop.1/hmac} \sfrdefinition{fcs_cop.1.1/hmac}{The TSF shall
  perform \ppassigned{HMAC value generation and verification} in accordance with
  a specified cryptographic algorithm \ppassigned{HMAC with \stdeleted{SHA-1,}
  \stassigned[list of SHA-2 Algorithms with 256bit size or more]{SHA-256}} and
  cryptographic key sizes \stassigned[cryptographic key sizes]{160 and 256 bit}
  that meet the following: \ppassigned{FIPS PUB 180-4 \autocite{FIPS180-4},
    RFC~2404 \autocite{rfc2404}, RFC~4868 \autocite{rfc4868}, RFC~5996
    \autocite{rfc5996}}.}



\sfrsubsection{fcs_ckm.1}

\sfrdefinition{fcs_ckm.1.1}{The TSF shall generate cryptographic keys in
  accordance with a specified cryptographic key generation algorithm
  \stassigned[cryptographic key generation algorithm]{PRF-HMAC-SHA256} and
  specified cryptographic key sizes \stassigned[cryptographic key
  sizes]{256~bit} that meet the following: \ppassigned{TR-03116
    \autocite{TR03116}}.\par
  \strefined{The following algorithms and preferences are supported for TLS key
    negotiation}
  \begin{sfritemize}
  \item \strefined{Diffie-Hellman Group 14 according to \rfc{3526}
      \cite{rfc3526} for key establishment during TLS}
  \item \strefined{DH exponent shall have a minimum length of 384
      bits}
  \item \strefined{Forward secrecy shall be provided}
  \item \strefined{Ephemeral elliptic curve DH key exchange supports
      the \mbox{P-256} and the \mbox{P-384} curves according to FIPS186-4
      \cite{FIPS186-4} as well as the brainpoolP256r1 and the
      brainpoolP384r1 curves according to \rfc{5639} and \rfc{7027}
      \cite{rfc5639,rfc7027}}
  \item \strefined{Peer authentication (if required): X.509
      certificate with RSA 2048 bit keys}
  \end{sfritemize}
}


\sfrsubsection{fcs_ckm.2/ike}
\sfrdefinition{fcs_ckm.2.1/ike}{\sfrunmodifiedfrompp{6.2.7}}

\sfrsubsection{fcs_ckm.2/tls}
\sfrdefinition{fcs_ckm.2.1/tls}{\sfrunmodifiedfrompp{6.2.7}}

\sfrsubsection{fcs_ckm.4}

\sfrdefinition{fcs_ckm.4.1}{The TSF shall destroy cryptographic
  keys in accordance with a specified cryptographic key destruction
  method \stassigned[cryptographic key destruction method]{by
    overwriting with zeros} that meets the following: \stassigned[list
  of standards]{none}.}


\hrefsubsection{sfr.def.cryptotls}{TLS-Channels With Secure Cyptographic Algorithms}

\sfrsubsection{ftp_itc.1/tls}
\sfrdefinition{ftp_itc.1.1/tls}{\sfrunmodifiedfrompp{6.2.8}}
\sfrdefinition{ftp_itc.1.2/tls}{\sfrunmodifiedfrompp{6.2.8}}

\sfrdefinition{ftp_itc.1.3/tls}{The TSF shall initiate
  communication via the trusted channel for \ppassigned{communication
    required by the administration interface} \stassigned[list of other
  functions for which a trusted channel is required]{any
    connection specified in Table~\ref{tab:tlsconnections}.}}

\applicationnote{\sfr{ftp_itc.1/tls}}{\label{appnote:brainpool} The TOE supports
  key exchange with elliptic curves.The supported curves are list in
  \tableref{tab:elliptic-curves}.}


\sfrsubsection{fpt_tdc.1/tls.zert}

\sfrdefinition{fpt_tdc.1.1/tls.zert}{The TSF shall provide the
  capability to consistently interpret
  
  \begin{sfrenumeration}
  \item \ppassigned{X.509 certificates for TLS connections}
  \item \ppassigned{Revocation information for certificates for TLS connections received via OCSP.}
  \item \stassigned[additional list of data types]{no other data types}
  \end{sfrenumeration}

  when shared between the TSF and another trusted IT product.}

 
\sfrdefinition{fpt_tdc.1.2/tls.zert}{The TSF shall use
  \ppselected{interpretation rules} when interpreting the TSF data from another
  trusted IT product.}

\sfrsubsection{fcs_cop.1/tls.aes}

\sfrdefinition{fcs_cop.1.1/tls.aes}{\sfrunmodifiedfrompp{6.2.8}}

\sfrsubsection{fcs_cop.1/tls.auth}

\sfrdefinition{fcs_cop.1.1/tls.auth}{\sfrunmodifiedfrompp{6.2.8}}


\hrefsubsection{sfr.def.additional}{Additional Requirements}

This section contains security requirements that are defined in addition to the
Protection Profile. The requirements are extended by the component
\sfr{fcs_rng.1/hashdrbg} defined in \chapterref{comp.pp-fcsrng}.

\sfrsubsection{fcs_rng.1/hashdrbg}

\hierarchicalto{No other components}
\dependencies{No dependencies}

\sfrdefinition{fcs_rng.1.1/hashdrbg}{The TSF shall provide a
  \stselected[physical, non-physical true, deterministic, hybrid
  physical, hybrid deterministic]{deterministic} random number
  generator that implements: \stassigned[list of security
  capabilities]{}
  \begin{sfrenumeration}
  \item[\stassigned{(1)}] \stassigned{If initialized with a random
      seed using PTRNG of class PTG.2 as random source, the internal
      state of the RNG shall have at least 100 bits min-entropy.}
  \item[\stassigned{(2)}] \stassigned{The RNG provides forward
      secrecy.}
  \item[\stassigned{(3)}] \stassigned{The RNG provides backward
      secrecy even if the current internal state is known.}
  \end{sfrenumeration}
}

\sfrdefinition{fcs_rng.1.2/hashdrbg}{The TSF shall provide random numbers that
  meet: \stassigned[a defined quality metric]{}
  \begin{sfrenumeration}
  \item[\stassigned{(1)}] \stassigned{The RNG gets initialized during
      every startup and after 2048 requests with a random seed of
      minimal 384 bits using a PTRNG of class PTG.2. The RNG generates
      output for which more than $2^{34}$ strings of bit length 128
      are mutually different with probability $w > 1- 2^{(-16)}$.}
  \item[\stassigned{(2)}] \stassigned{Statistical test suites cannot
      practically distinguish the random numbers from output sequences
      of an ideal RNG. The random numbers must pass test procedure A.}
  \end{sfrenumeration}}


% !TEX root = ase
%%% Local Variables:
%%% mode: latex
%%% TeX-master: "../ase"
%%% TeX-engine: luatex
%%% End:




\hrefsection{sfr.toe.eal}{Sicherheitsanforderungen an die
  Vertrauenswürdigkeit des EVG}

Die Sicherheitsanforderungen an die Vertrauenswürdigkeit für dieses Security
Target entsprechen denen, die in \citepp{} definiert sind.

\hrefsection{sfr.ratio}{Erklärung der Sicherheitsanforderungen}

\hrefsubsection{sfr.ratio.dep}{Erklärung der Abhängigkeiten der SFR}

Die Abhängigkeiten der in \sectref{sfr.def} aufgestellten funktionalen
Sicherheitsanforderungen sind erfüllt. Es gelten dieselben Auflösungen von
Abhängigkeiten, wie sie im Schutzprofil \citepp[Abschnitt~6.4.2]{} beschrieben
sind.

Die Abhängigkeiten der über das Schutzprofil hinaus aufgenommenen
Sicherheitsanforderungen sind bei der Definition des jeweiligen SFR
notiert.  Die zusätzlich aufgenommenen SFR sind Iterationen
bestehender Komponenten, sodass sich durch diese keine neuen
Abhängigkeiten ergeben.

Die in \sectref{comp.pp-fcsrng} neu eingeführte Komponente
\secitem{FCS_RNG.1} hat keine Abhängigkeiten, die aufgelöst werden
müssen.

Die Darstellungen in \tableref{tab:sfr.subjobj2sfr} und
\tableref{tab:sfr.sfr2subjobj} zeigen, wie die zusätzlich angenommenen Subjekte
aus \tableref{tab:sfr.intro.model} in Zusammenhang zu den
Sicherheitsanforderungen stehen.

\begin{longtable}[c]{@{}l>{\centering}*{17}{c}<{\centering}@{}}
  \toprule
  \directlua{print_table_header('mainsfr', 'sfrlinknoindex')} \\
  % 
  \midrule \endhead
  \midrule \caption{Abbildung der Subjekte auf SFR}  \endfoot
  \bottomrule \caption{Abbildung der Subjekte auf SFR} \label{tab:sfr.subjobj2sfr} \endlastfoot
  \directlua{print_table_body('subjobj', 'mainsfr', 'subjobjlink', cc_core.getSubjobj2Sfr)}
\end{longtable}

\begin{longtable}[c]{@{}l>{\centering}*{2}{c}<{\centering}@{}}
  \toprule
  \directlua{print_table_header('subjobj', 'subjobjlink')} \\
  % 
  \midrule \endhead
  \midrule \caption{Abbildung der SFR auf Subjekte}  \endfoot
  \bottomrule \caption{Abbildung der SFR auf Subjekte} \label{tab:sfr.sfr2subjobj} \endlastfoot
  \directlua{print_table_body('mainsfr', 'subjobj', 'sfrlinknoindex', cc_core.getSfr2Subjobj)}
\end{longtable}



\hrefsubsection{sfr.ratio.overview}{Überblick der Abdeckung von
  Sicherheitszielen}


Die Zuordnung von Sicherheitszielen zu Sicherheitsanforderungen entspricht
weitestgehend der Zuordnung, die in \citepp{} getroffen
wurde. \tableref{tab:sfr.ratio.sfrtoobjmapping} zeigt den Zusammenhang.

\afterpage{%
  \clearpage% Flush earlier floats (otherwise order might not be correct)
  \centering % Center table
  \begin{longtable}[c]{@{}l*{13}{T}@{}}
  \toprule
  \directlua{print_table_header('obj_no_env', 'objlink')}
  \\
  % 
  \midrule \endhead
  \bottomrule \endfoot
  \bottomrule  \caption*{Mapping of objectives to SFR} \endfoot
  \bottomrule  \caption{Mapping of objectives to SFR} \label{tab:sfr.ratio.sfrtoobjmapping} \endlastfoot
  \directlua{print_table_body('mainsfr', 'obj_no_env', 'sfrlinknoindex', cc_core.getSfr2Obj)}
\end{longtable}

% !TEX root = ../ase
%%% Local Variables:
%%% mode: latex
%%% TeX-master: "../ase"
%%% TeX-engine: luatex
%%% End:
%     
  % \clearpage% Flush page
}

\hrefsubsection{sfr.ratio.obj}{Detaillierte Erklärung für die
  Sicherheitsziele}

Die detaillierte Erklärung der Sicherheitsziele wird unverändert aus \citepp{}
übernommen.

\hrefsection{sfr.ratio.eal}{Erklärung für die gewählte EAL-Stufe}

Die Erklärung der gewählten EAL-Stufe wird unverändert aus dem Schutzprofil
\citepp{} übernommen.


%%% Local Variables:
%%% mode: latex
%%% TeX-master: "ase"
%%% TeX-engine: luatex
%%% End:
