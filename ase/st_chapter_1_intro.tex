\hrefchapter{intro}{Einführung in das Security Target}

Der TOE, der in diesem Dokument beschrieben wird, ist der \emph{\thisTOE{}}. Der
TOE ist eine sichere Komponente, die als \thisproduct{} eingesetzt wird.

Dieses Dokument ist das \emph{Security Target}, in dem die funktionalen und
organisatorischen Sicherheitsanforderungen des TOE und seiner Einsatzumgebung
beschrieben werden.  Dieses Dokument findet seine formale Grundlage in:

\begin{itemize}
\item \citetitle{\thispp} \autocite{\thispp}
\end{itemize}

\hrefsection{intro.refst}{ST Referenz}

\begin{tabularx}{1\textwidth}{@{}p{0.3\textwidth}X@{}}
  \toprule
  Titel des Dokuments & Security Target / \thisproduct\\
  Version des Dokuments & \documentversion{\thisdocument}\\
  Datum des Dokuments & \documentdate{\thisdocument}\\
  Allgemeiner Status: & \\
  Autor  &  \thisdeveloper{}\\
  Editor &  \\
  \bottomrule
\end{tabularx}

\hrefsection{intro.reftoe}{TOE Referenz}

\begin{tabularx}{1\textwidth}{@{}p{0.3\textwidth}X@{}}
  \toprule
  Evaluierungsgegenstand & \thisTOE{}\\
  Version des EVG & \toeversion{} \\
  Hersteller & \thisdeveloper{} \\
  Vertrauenswürdigkeitsstufe &
  \secitemformat{EAL3} erweitert um
  \secitemformat{AVA\_VAN.3}, \secitemformat{ADV\_IMP.1}, %
  \secitemformat{ADV\_TDS.3}, \secitemformat{ADV\_FSP.4}, %
  \secitemformat{ALC\_TAT.1}, and \secitemformat{ALC\_FLR.2} %
  (Kurzbezeichnung „\secitemformat{EAL3+}“)\\
  CC Version &  3.1 Release 5\\
  \bottomrule
\end{tabularx}

\cleardoublepage{}

\hrefsection{intro.overview}{Überblick über den TOE}

Der Evaluierungsgegenstand ist der \thisTOE{}.

Der Lieferumfang des TOE umfasst ebenfalls die Betriebsdokumentation für
\thisproduct{}. Somit entspricht der TOE dem im Schutzprofil \autocite{\thispp}
genannten Umfang und Aufbau.

\hrefsubsection{intro.overview.toetype}{TOE Typ}

\thisproduct{} implementiert -- konform zu \citepp{} -- den Produkttyp eines
VPN-Routers.

\hrefsubsection{intro.overview.usage}{Verwendung und Hauptfunktionalität des
  TOE}

\hrefsubsection{intro.overview.nontoe}{Erforderliche Non-TOE
  Hardware/Software/Firmware}


\hrefsection{intro.desc}{Beschreibung des TOE}

\hrefsubsection{intro.desc.goals}{Hauptziele des TOE}

\hrefsubsection{intro.desc.aufbau}{Aufbau des TOE}

% \begin{figure}[tb]
%   \centering{}
%   \hypertarget{fig:intro.desc.case}{%
%     \includegraphics[width=0.8\columnwidth,keepaspectratio]%
%     {../common/media/thetoe_front_1.jpg}
%     \includegraphics[width=0.8\columnwidth,keepaspectratio]%
%     {../common/media/thetoe_back_1.jpg}
%     \caption{Abbildung des \thisproduct}
%     \label{fig:intro.desc.case}}
% \end{figure}

Das Betriebssystem des \thisproduct{} ist GNU/Linux. Teile des Betriebssystems
setzen Sicherheitsanforderungen an den TOE um und sind somit SFR-enforcing.

\hrefsubsection{intro.desc.env}{Einsatzumgebung des TOE}

\hrefsubsection{intro.desc.hardware}{Hardware des \thisproduct}

\hrefsubsection{intro.desc.intf}{Schnittstellen des \thisproduct}

\hrefsubsubsection{intro.desc.intf.phys}{Physische Schnittstellen}

Alle Schnittstellen des \thisproduct{} sind physisch am Gehäuse des Geräts
untergebracht. Die außen sichtbaren Schnittstellen sind auf dem Foto des TOE in
\figureref{fig:intro.desc.case} zu erkennen.

\begin{description}
\item[\intf{PS.LAN}] ist die Schnittstelle ins LAN\dots
\item[\intf{PS.WAN}] ist die Schnittstelle ins WAN\dots
\item[\intf{PS.LED}] repräsentiert die LEDs an der Außenseite des Geräts.
\end{description}

\hrefsubsubsection{intro.desc.intf.log}{Logische Schnittstellen}

The TOE provides the logical interfaces described in the protection profile \autocite{\thispp}. They are repeated here.

\begin{description}

\item[\hypertarget{tsfi.ls.lan}{\intf{LS.LAN}}] is the TOE's interface to the
  local area network of the operating environment. In addition to the interfaces
  described in the protection profile, there are further protocol specific
  interfaces described here. \tableref{tab:tsfi.ls.lan} lists these logical
  interfaces.

\item[\hypertarget{tsfi.ls.wan}{\intf{LS.WAN}}] is the TOE's interface to the
  wide area network of the operating environment, the Internet. In addition to
  the interfaces described in the protection profile, there are further protocol
  specific interfaces described here. \tableref{tab:tsfi.ls.wan} lists these
  logical interfaces.
 
\item[\hypertarget{tsfi.ls.led}{\intf{LS.LED}}] represents the logical interface
  to the display and the buttons of \intf{PS.LED}.

\end{description}

  \begin{table}[htbp]
    \centering
    \begin{tabularx}{1\columnwidth}{@{}llX@{}}
      \toprule
      Label & Client/Server & Purpose of the interface \\
      \midrule
      \hypertarget{tsfi.ls.lan.ether}{\tsfi{ls.lan.ether}} & --- & media access \\
      \hypertarget{tsfi.ls.lan.ip}{\tsfi{ls.lan.ip}} & --- & access to the Internet layer \\
      \hypertarget{tsfi.ls.lan.tcp}{\tsfi{ls.lan.tcp}} & --- & access to the transport layer \\
      \hypertarget{tsfi.ls.lan.tls}{\tsfi{ls.lan.tls}} & server & transport security with TLS~1.2 \\
      \hypertarget{tsfi.ls.lan.udp}{\tsfi{ls.lan.udp}} & --- & access to the transport layer \\
      \hypertarget{tsfi.ls.lan.httpmgmt}{\tsfi{ls.lan.httpmgmt}} & server & HTTP access to the management console \\
      \bottomrule
    \end{tabularx}
    \caption{Logical Interfaces on \tsfi{ls.lan}}
    \label{tab:tsfi.ls.lan}
  \end{table}

  \begin{table}[htbp]
    \centering
    \begin{tabularx}{1\columnwidth}{@{}llX@{}}
      \toprule
      Label & Client/Server & Purpose of the interface \\
      \midrule
      \hypertarget{tsfi.ls.wan.ether}{\tsfi{ls.wan.ether}} & --- & media access \\
      \hypertarget{tsfi.ls.wan.ip}{\tsfi{ls.wan.ip}} & --- & access to the Internet layer \\
      \hypertarget{tsfi.ls.wan.tcp}{\tsfi{ls.wan.tcp}} & --- & access to the transport layer \\
      \hypertarget{tsfi.ls.wan.ntp}{\tsfi{ls.wan.ntp}} & client & Obtaining time \\
      \hypertarget{tsfi.ls.wan.dhcp}{\tsfi{ls.wan.dhcp}} & client & Obtaining IP addresses in the WAN \\
      \hypertarget{tsfi.ls.wan.udp}{\tsfi{ls.wan.udp}} & --- & access to the transport layer \\
      \hypertarget{tsfi.ls.wan.ipsec}{\tsfi{ls.wan.ipsec}} & --- & VPN data traffic\\
      \bottomrule
    \end{tabularx}
    \caption{Logical Interfaces on \tsfi{ls.wan}}
    \label{tab:tsfi.ls.wan}
  \end{table}


%%% Local Variables:
%%% mode: latex
%%% TeX-master: t
%%% End:
 % Wird auch in anderen Dokumenten verwendet

\clearpage

\hrefsubsection{intro.desc.physdiff}{Aufbau und physische Abgrenzung des TOE}

Der TOE besteht aus folgenden Subsystemen:

\begin{description}
\item[\tds{sub.vpn}] enthält die VPN-Funktionen wie IPSec und IKE.
\item[\tds{sub.ntpclient}] synchronisiert die Systemzeit mit einem Zeitserver.
\item[\tds{sub.selfprotect}] enthält Schutzmechanismen für den TOE.
\item[\tds{sub.adminsystem}] wird für die Administration des TOE verwendet.
\item[\tds{sub.cryptsystem}] bietet kryptografische Basisfunktionen.
\item[\tds{sub.tls}] erstellt TLS-Verbindungen für die Administrationsschnittstelle.
\end{description}



\hrefsubsection{intro.desc.secservices}{Logische Abgrenzung: Vom TOE
  erbrachte Sicherheitsdienste}

\hrefsubsection{intro.desc.scope}{Physischer Umfang des TOE}

Der physische Umfang des TOE umfasst die in \tableref{tab:intro.desc.scope} aufgelisteten Komponenten.


\renewcommand{\arraystretch}{1.5}
\begin{table}[htb]
  \centering
  \begin{tabularx}{0.95\linewidth}{@{}p{7cm}Xl@{}}
    \toprule
    Komponente & Beschreibung & Version\\ \midrule
    Firmware Image & Die Firmware  des TOE & \toeversion{} \\
    Guidance Documentation („Administrationshandbuch“) & Die Guidance Documentation beschreibt die sichere Verwendung des TOE & \toeversion{} \\
    Benutzerhandbuch ("`Allgemeine Gebrauchsanleitung \thisproduct{}"') & Das Benutzerhandbuch beschreibt die allgemeine Verwendung, sowohl dessen TOE Anteile als auch die nicht-TOE Anteile & \toeversion\\
    \bottomrule
  \end{tabularx}
  \caption{Physischer Umfang des TOE}
  \label{tab:intro.desc.scope}
\end{table}
\renewcommand{\arraystretch}{1.0}

%%% Local Variables:
%%% mode: latex
%%% TeX-master: "../ase"
%%% End:
