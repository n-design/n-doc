Der TOE verfügt über die logischen Schnittstellen, die das Schutzprofil
\autocite{\thispp} beschreibt. Diese werden hier der besseren Lesbarkeit halber
wiederholt.

\begin{description}

\item[\hypertarget{tsfi.ls.lan}{\intf{LS.LAN}}] ist die Schnittstelle des TOE
  ins lokale Netzwerk der Einsatzumgebung. Zusätzlich zu den im Schutzprofil
  genannten Schnittstellen werden hier weitere protokollspezifische
  Schnittstellen definiert. \tableref{tab:tsfi.ls.lan} listet diese logischen
  Schnittstellen.

\item[\hypertarget{tsfi.ls.wan}{\intf{LS.WAN}}] ist die Schnittstelle des TOE
  zum WA. Verschiedene Protokolle implementieren weitere Logische Schnittstellen
  in Richtung des WAN. \tableref{tab:tsfi.ls.wan} listet diese logischen
  Schnittstellen.
 
\item[\hypertarget{tsfi.ls.led}{\intf{LS.LED}}] repräsentiert die
  logische Schnittstelle zum Display und den Bedienknöpfen über
  \intf{PS.LED}.

\end{description}

  \begin{table}[htbp]
    \centering
    \begin{tabularx}{1\columnwidth}{@{}llX@{}}
      \toprule
      Bezeichner & Rolle & Zweck der Schnittstelle \\
      \midrule
      \hypertarget{tsfi.ls.lan.ether}{\tsfi{ls.lan.ether}} & --- & Protokoll auf Zugangsschicht \\
      \hypertarget{tsfi.ls.lan.ip}{\tsfi{ls.lan.ip}} & --- & Zugang zur Internet-Schicht \\
      \hypertarget{tsfi.ls.lan.tcp}{\tsfi{ls.lan.tcp}} & --- & Zugang zur Transportschicht \\
      \hypertarget{tsfi.ls.lan.tls}{\tsfi{ls.lan.tls}} & beide & Sicherung der Verbindung mit TLS~1.2 \\
      \hypertarget{tsfi.ls.lan.udp}{\tsfi{ls.lan.udp}} & --- & Zugang zur Transportschicht \\
      \hypertarget{tsfi.ls.lan.httpmgmt}{\tsfi{ls.lan.httpmgmt}} & Server & HTTP-Zugriff auf Managementschnittstelle \\
      \bottomrule
    \end{tabularx}
    \caption{Logische Schnittstellen an \tsfi{ls.lan}}
    \label{tab:tsfi.ls.lan}
  \end{table}

  \begin{table}[htbp]
    \centering
    \begin{tabularx}{1\columnwidth}{@{}llX@{}}
      \toprule
      Bezeichner & Rolle & Zweck der Schnittstelle \\
      \midrule
      \hypertarget{tsfi.ls.wan.ether}{\tsfi{ls.wan.ether}} & --- & Protokoll auf Zugangsschicht \\
      \hypertarget{tsfi.ls.wan.ip}{\tsfi{ls.wan.ip}} & --- & Zugang zur Internet-Schicht \\
      \hypertarget{tsfi.ls.wan.ntp}{\tsfi{ls.wan.ntp}} & Client & Abruf der Uhrzeit \\
      \hypertarget{tsfi.ls.wan.dhcp}{\tsfi{ls.wan.dhcp}} & Client & Adressbezug im WAN \\
      \hypertarget{tsfi.ls.wan.tcp}{\tsfi{ls.wan.tcp}} & --- & Zugang zur Transportschicht \\
      \hypertarget{tsfi.ls.wan.udp}{\tsfi{ls.wan.udp}} & --- & Zugang zur Transportschicht \\
      \hypertarget{tsfi.ls.wan.ipsec}{\tsfi{ls.wan.ipsec}} & --- & VPN Datenverkehr \\
      \bottomrule
    \end{tabularx}
    \caption{Logische Schnittstellen an \tsfi{ls.wan}}
    \label{tab:tsfi.ls.wan}
  \end{table}


%%% Local Variables:
%%% mode: latex
%%% TeX-master: t
%%% End:
